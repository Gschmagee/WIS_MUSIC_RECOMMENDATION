\documentclass[twosided,a4,10pt]{article}
\usepackage[utf8]{inputenc}
\usepackage{amsmath}
\usepackage{amsfonts}
\usepackage{amssymb}
\usepackage{textcomp}
\usepackage{german}
\usepackage{graphicx}
\usepackage[usenames,dvipsnames]{xcolor}
\usepackage{pifont}
\usepackage{nicefrac}
\usepackage{sectsty}
\usepackage{dblfnote}
% ------
% Fonts and typesetting settings
\usepackage[sc]{mathpazo}
\usepackage[T1]{fontenc}
\linespread{1.1} % Palatino needs more space between lines
\usepackage{microtype}
\subsectionfont{\fontsize{10}{15}\selectfont}

% ------
% Page layout
\usepackage[hmarginratio=1:1,top=32mm,columnsep=20pt]{geometry}
\usepackage[font=it]{caption}
\usepackage{paralist}
\usepackage{multicol}

%------
%caption hack
\usepackage{caption}

\DeclareCaptionType{faltung}[][List of equations]
\captionsetup[faltung]{labelformat=empty}

% ------
% Abstract
\usepackage{abstract}
\renewcommand{\abstractnamefont}{\normalfont\bfseries}
\renewcommand{\abstracttextfont}{\normalfont\small\itshape}


% ------
% Titling (section/subsection)
\usepackage{titlesec}
%\renewcommand\thesection{\Roman{section}}
\titleformat{\section}[block]{\large\scshape\centering}{\thesection.}{1em}{}

% ------
% Clickable URLs (optional)
\usepackage[hyphens]{url}
\usepackage{hyperref}

% ------
% Header/footer
\usepackage{fancyhdr}
\pagestyle{fancy}
%	\fancyhead{}
%	\fancyfoot[C]{WIS WS 2017/18 $\cdot$
% Software Engineering $\cdot$ Prof. Dr. Dünnweber}
\fancyhead[R]{OTH Regensburg $\cdot$ Fakultät IM}
%	\fancyfoot[RO,LE]{\thepage}
\fancyfoot[L]{WIS $\cdot$ WS 2017/18}
\fancyfoot[R]{Prof. Dr. Dünnweber}
\fancyfoot[C]{\thepage}


% ------
% Maketitle metadata
\title{\vspace{-5mm}%
	\fontsize{20pt}{10pt}\selectfont
	\textbf{Inhaltsbasierte Musikempfehlung mit Convolutional Neuronalen Netzwerken}
}	
\vspace{-5mm}\date{}
\author{
	\large\begin{minipage}[t]{0.5\linewidth}
		\begin{center}
			\textsc{Weidhas Philipp}\\[2mm]
			\normalsize	Matr.nr: 123456\\
			\normalsize
			\href{mailto:philipp.weidhas@st.oth-regensburg.de}
			{philipp.weidhas@st.oth-regensburg.de}
		\end{center}
	\end{minipage}
	\begin{minipage}[t]{0.5\linewidth}
		\begin{center}
			\textsc{Wildgruber Markus}\\[2mm]
			\normalsize	Matr.nr: 123456\\
			\normalsize
			\href{mailto:markus.wildgruber@stud.oth-regensburg.de}
			{markus.wildgruber@stud.oth-regensburg.de}
		\end{center}
	\end{minipage}
}




%%%%%%%%%%%%%%%%%%%%%%%%
\begin{document}
	
	\maketitle
	\thispagestyle{fancy}
	
	
	
	\begin{multicols}{2}
		
		\begin{abstract}
			\noindent Hier kommt die Zusammenfassung...
		\end{abstract}
		
		
		\section{Einleitung}
		
		Im ersten Halbjahr des Jahres 2017 wurden 62\% der Einnahmen der amerikanischen Musikindustrie durch Streaming Plattformen\footnote[1]{wie Spotify, Apple Music, Pandora etc.} erzielt. Im Vergleich zu Vorjahr erhöhten sich dadurch die Einnahmen um 48\% auf 2.5\$ Milliarde \cite{friedlander}. Dieser Erfolg basiert nicht nur auf einer guten Verfügbarkeit der Lieder und einem günstigen Preis sondern auch auf automatischen Musikempfehlungsdiensten, welche dem Nutzer ein angenehmeres Konsumverhalten ermöglichen.\newline
		Obwohl Empfehlungsdienste in den letzten Jahren viel erforscht wurden, ist das Problem der Musikempfehlung sehr komplex. Neben einer große Anzahl an verschiedenen Stile und Genres, beeinflussen sowohl soziales- und geographisches Umfeld, sowie der aktuelle Gemütszustandes die Vorliebe eines Hörers. \cite{oord}\newline
		TO DO \newline\\
		
		In der Musik Information Retrieval (MIR) gibt es vier Kategorien \cite{schedl} die einen Einfluß auf die Wahrnehmung von ähnlicher Musik haben.\newline \textit{Musikmerkmale} sind Eigenschaften, welche aus dem Audiosignal eines Liedes extrahiert werden. Dazu zählen Aspekte wie der Rhythmus, die Melodie, die Harmonie oder die Stimmung eines Stücks.\newline
		Als \textit{Musikkontext} versteht man alle Aspekte, die nicht aus dem Audiosignal abgeleitet werden, sondern Informationen die über ein Musikstück bekannt sind. Beispielsweise Metadaten wie der Titel eines Lieds, das Genre, Name des Künstlers oder das Erscheinungsjahr.\newline
		Die \textit{Benutzereigenschaften} beziehen sie auf Persönlichkeitsmerkmale, wie Geschmack, musikalisches Wissen und Erfahrung oder den demographischen Hintergrund.\newline
		Im Unterschied dazu steht der \textit{Benutzerkontext}, der sich auf die aktuelle Situation des Hörers bezieht. Dabei wird er durch seine Umgebung, seiner Stimmung oder der aktuellen Aktivität beeinflußt. \cite{knees}\newline\\
		Bislang werden Informationen über den Hörer durch ein Benutzerprofil repräsentiert. Das Profil enthält nur wenig Hintergrund Informationen des Hörers und beschränkt sich auf Lieder, die ein Benutzer angehört und bewertet hat \cite{oord}. Das Nutzen dieser Daten, um Musikvorschläge abzugeben wird als kollaboratives Filtern (KF) bezeichnet. In der Studie von Vigliensoni und Fujinaga \cite{vigliensoni} zeigt sich ein deutlicher Unterschied zwischen herkömmlichen Benutzerprofilen und das Einfügen von Zusatzinformationen. Durch das Hinzufügen der Features demographischen Hintergrund und Entdeckergeist des Hörers konnte im Vergleich zu einem herkömmlichen Profil eine 12\% besser Genauigkeit erreicht werden.\newline\\
		Der weitere Verlauf der wissenschaftlichen Arbeit ist wie folgt organisiert. Im 2. Abschnitt werden verschieden Ansätze in den jeweiligen Methodenbereichen vorgestellt. Im 3. Kapitel werden die erfolgreichsten Ansätze miteinander verglichen. Teil 4 zeigt ein eigenes Experiment zu dem Thema. Abschnitt 5 schließt diese Arbeit ab und diskutiert zukünftige Forschungsrichtungen. //TO DO
		
		\section{Methoden zur Musikempfehlung}
		Es gibt verschiedene Methoden, die in Musikempfehlungssystemen verwendet werden: kollaboratives -, merkmalsbasiertes -, kontext-basiertes Filtern und die hybride Methode. Diese werden genutzt, um Informationen aus der in der Einleitung genannten Eigenschaften zu gewinnen und diese für Empfehlungen an den Nutzer zu verarbeiten. \cite{wang}
	
		\subsection{Kollaborativer Filter}
		Kollaboratives Filtern prognostiziert Vorlieben eines Hörers, indem es aus unterschiedlichen Benutzer-Lied Verhältnissen lernt. Es basiert auf der Annahme, dass Verhalten und Bewertungen andere Nutzer auf eine vernünftige Vorhersage für den aktiven Benutzer schließen lassen \cite{celma}. Durch explizite\footnote[2]{ Bewertungen eines Nutzers} und implizite\footnote[3]{Beobachten des Konsumverhalten} Rückmeldung eines Hörers an das Empfehlungssystem empfiehlt dieses neue Lieder, indem es Gemeinsamkeiten auf Basis der Bewertungen vergleicht \cite{mcfee}.\newline
		Im diesen Verfahren wird der Ansatz verfolgt, dass Lieder einem Nutzer auf Grundlage von Nutzungsverhalten anderer Anwender der gleichen Plattform vorgeschlagen werden. In der praktischen Umsetzung bedeutet dies: hört ein Anwender ein bestimmtes Musikstück, werden ihm von der Empfehlungsplattform, Lieder vorgeschlagen welche Nutzer in Zeitraum zuvor nach diesem Stück hörten. Dieses Verfahren geht davon aus, dass durch die Verbindung der Lieder durch vorhergehende Aufrufe eine gute Aussage darüber getroffen werden kann wie gut diese Stücke zusammen passen. Werden Lieder häufig nacheinander gehört, wird diese Verbindung höher bewertet und die Empfehlung häufiger ausgesprochen. Auch wird das Verhalten und der Musikgeschmack des Kunden selbst durch ein System analysiert, um so über Ähnlichkeiten der Kundenpräferenzen mit derer anderer, diesen wiederum bessere Empfehlungen aussprechen zu können. So werden Lieder einem Musikstil zugeordnet und so zielgerichtet dem Nutzer nahegelegt.\newline
		Verschiedene Studien (\cite{mcfee}\cite{barrington}) zeigen, dass KF alternative Methoden in der Genauigkeit übertrifft, weshalb es nicht nur im Bereich der Musikempfehlung als die erfolgreichste gilt.\newline
		Trotz der Popularität des KF gibt es Probleme, die bei der Verwendung dieser Methode beachtet werden müssen. Das Cold-Start Problem besteht darin, dass noch keine Bewertungen für ein Lied vorliegen, wodurch es auch nicht vorgeschlagen werden kann. Dasselbe Problem gibt es bei einem neuen Benutzer: diesem kann kein guter Vorschlag gemacht werden, da es an Information mangelt welche Art von Musik ihm gefällt. Neben dem Cold-Start Problem gibt es noch weitere Probleme. \cite{celma} 
		
		\subsection{Merkmalsbasierter Filter}
		%Als erstes wird nun ein genauerer Blick auf den inhaltsbezogenen Ansatz geworfen. Mittels diesem Verfahrens werden Nutze Musikstücke aufgrund aus Lieder gewonnener Informationen vorgeschlagen. Dies bedeutet im Detail dass aus den Musikstücken mittels verschiedenster Metriken die Audio Signale eines Liedes analysiert werden um Erkenntnisse über die Stimmung eines Musikstücks, die Frequenz oder Rhythmus zu erhalten. Auf Grund dieser Informationen können Stücke dem Konsumenten vorgeschlagen werden die einen gleichen oder sehr ähnlichen Inhalt bieten.
		\subsection{Kontextbasierter Filter}

				
		\subsection{Hybride Methode}
		Bei hybriden Methoden werden kollaborative, merkmalsbasierte und kontextbasierter Filter miteinander verknüpft, wodurch ein besseres Empfehlungsergebnis mit weniger Nachteilen der einzelnen Methode zu erzielen. Meistens wird ein kollaborativer Filter mit einem der beiden anderem kombiniert.\newline
		Als \textit{gewichtet} wird eine hybride Methode bezeichnet, bei der Empfehlungswerte der einzelnen Methoden durch eine Linearkombination zusammengerechnet wird. Das Ergebnis der Linearkombination stellt den Empfehlungswertes eines Liedes dar. Durch unterschiedliche Gewichtung der Methoden kann das Empfehlungsergebnis optimiert werden. Der \textit{wechselnde} Ansatz benutzt ein bestimmtes Kriterium anhand dessen es die Methode zur Vorschlagsbestimmung wechselt. Dies kann beispielsweise dann der Fall sein, wenn der erste Filter kein zuverlässiges Ergebnis\footnote[4]{semantische Unterschiede} liefert. Dann wechselt das System den Filter und kann ein besseres Empfehlungsergebnis bekommen. Bei \textit{gemischten} hybriden Empfehlungen werden unterschiedliche Techniken direkt miteinander vermischt. Dadurch kann für ein System mit inhaltsbasierten Filter das Cold-Start Problem vermieden werden.\newline // TODO
		Hybride Methoden können einige Nachteile von kollaborativen Filtern entfernen. Allerdings stehen auch sie vor dem "Neuen Benutzer" Problem. Dennoch sind hybride Methoden sehr beliebt, da Information über einen neuen Benutzer schnell herausgefunden werden oder durch Profilangaben bereits nach der Registrierung vorhanden sind. \cite{burke}
		
		\section{CNN für Audiosignale}
		CNN sind durch das biologische Sehen inspiriert und konnten den ersten großen Erfolg im Bereich der Bildklassifizierung \cite{alex} verzeichnen. Trotzdem werden CNN auch in verschiedenen Audiobereich, wie der Spracherkennung \cite{graves} und der MIR mehr genutzt und erforscht.\newline
		In der MIR nutzen die ersten Forschungen CNNs, um die Aufgabe der Musikgenre-Klassifizierung \cite{lee} zu untersuchen. Die Ergebnisse\footnote[5]{richtigen Klassifizierung} zeigen, das eine automatisierte Klassifizierung die herkömmliche Methode MFCC deutlich übertriff. Das erste CNN für inhaltsbasierte Musikempfehlung \cite{oord} benutzt zunächst eine Matrix-Faktorisierung um Eigen Vektoren für alle Lieder zu erhalten. Anschließend wird  das Neuronale Netz für die Zuordnung der Audio-Inhalte zu den Eigen Vektoren genutzt. \cite{wang}\newline\\
		Im nachfolgenden Absatz wird der Aufbau, das Training und die Optimierung eines CNN beschrieben.
		
%Nachdem Alex Krizhevsky mit seinem Team den ImageNet ILSVRc-2012 Contest mit Hilfe eines CNN gewann. Wurden CNNs auch in anderen Bereichen neben der Bildklassifizierung \cite{alex} in Gesichtserkennung \cite{ding}, Spracherkennung \cite{graves} und der Inhaltsbasierten Musikempfehlung \cite{oord} mehr genutzt und erforscht.\newline
%Um diese unterschiedliche Funktionalität zu lernen, werden CNN mit drei verschiedenen Arten trainiert. Dem überwachten Lernen (supervised learning) bei dem das DNN eine Eingabe erhält, dessen Ausgabe bekannt ist. Durch das Vergleichen der Netzwerkausgabe mit der Erwarteten, kann das DNN dementsprechend konfiguriert werden. Beim Unbewachten Lernen (unsupervised learning) erhält das DNN verschiedene Eingaben und soll selbständig zusammenhänge zwischen diesen erkennen. Beim bestärkten Lernen (reinforcement learning) befindet sich das DNN in einer ihm unbekannter Umgebung, die es zu erforschen gilt. Gewünschtes Verhalten wird belohnt, wodurch es lernt die richtigen Entscheidungen zu treffen \cite{wang2}.\newline 
%Vor allem in den letzten Jahren hat sich das Convolutional Neuronales Netzwerk (CNN) als das erfolgversprechendste DNN erwiesen.\newline 
%Im folgenden Absatz wird eine Übersicht über den Aufbau, das Training und die Besonderheiten eines CNNs dargelegt. Anschließend werden verschiedene Ansätze der Inhaltsbasierten Musikempfehlung miteinander verglichen.
		
		\subsection{Convolutional Neuronalen Netze}
		Im Unterschied zu regulären DNN verwendet das CNN Neuronen, die drei Dimensionale angeordnete sind. Durch diese Anordnung ist es möglich größere Inputdaten in derselben Geschwindigkeit zu verarbeiten wie zuvor \cite{karpathy}. Um eine CNN Architektur zu erstellen werden drei Haupttypen von Schichten verwendet: Convolutional Layer (CL), Pooling Layer (PL) und ein Fully-Connected Layer (FCL).
		
		\subsubsection{Schichten eines Convolutional Neuronalen Netzwerks}
		
		\begin{minipage}{0.45\textwidth}
			\centering
			\includegraphics{img/faltung2.png}
			\captionof{figure}{Faltung einer 6x6 Matrix mit einem 3x3 Filter \cite{wikipic}}
			\label{img:faltung}
		\end{minipage}\newline
	
		\subsubsection*{Convolutional Layer}
		In einem CL findet eine Faltung der Eingangsdaten, in Form einer Matrix, und einem oder mehreren Filtern statt. Ein Filter dient beispielsweise zur Glättung oder zur Verkleinerung der Daten. Eine Verkleinerung der Eingangsmatrix findet statt, wenn ein Filter ohne Zero-padding\footnote[6]{Eine Matrix wird am Rand um Nullen erweitert.\\ Bsp. aus einer 7x7 Matrix wird eine 9x9 Matrix} verwendet wird. Die Parameter eines Filters werden zufällig initialisiert, können aber mit Hilfe eines Optimierungsverfahrens (3.1.3) angepasst werden. Werden mehrere Filter auf die Eingangsdaten angewendet, ändert sich die Tiefe der gesamten Ausgangsmatrix entsprechend der Anzahl der Filter. \cite{karpathy}\newline
		In Abbildung \ref{img:faltung} ist die Eingabematrix I eine 6x6 Matrix und K ein 3x3 Filter. Die Ausgabematrix S wir an den Stellen (i,j) durch die Gleichung (\ref{faltung1}) berechnet. Eine genauere Herleitung der Gleichung findet der Leser u. a. bei \cite{goodfellow}(328f).
		\begin{equation}\label{faltung1}
		S(i,j) =(I \star K)(i,j)
		\end{equation}
		\begin{equation}\label{faltung2}
		(I \star K)(i,j) =\newline\sum_{m}^{}\sum_{n}^{}I(i+m,j+n)K(m,n)
		\end{equation}\newline\\
		
		\subsubsection*{Pooling Layer}
		Ein PL wird zwischen zwei Cl eingefügt. Ihre Funktion besteht darin, die Größe der Daten zu reduzieren und damit die Anzahl der Parameter für das nächste CL. Durch die Reduzierung wird die Berechnung des gesamten Netzwerkes beschleunigt. \cite{karpathy}\newline Ein PL wandelt die Ausgabe eines CL, durch eine statistische Zusammenfassung von nebeneinander liegenden Ausgängen, um. Verschiedene Methoden für ein Pl sind: Max Pooling \cite{zhou}, eine Übergabe der größten Zahl in einem rechteckigen Umfeld; die Durchschnittsberechnung des Umfeldes oder ein gewichteter Durchschnitt basierend auf der Entfernung eines zentralen Punktes \cite{goodfellow}(355).\newline
		Abbildung \ref{img:pooling} zeigt einen 2x2 Max-Filter, der auf eine 4x4 Datenmatrix angewandt wird. Die Verschiebung oder Stride des Filters ist 2 dh. der Filter wird zunächst auf der y-Achse verschoben. Erreicht er dort das Ende wird er um eine Stride auf der x-Achse verschoben und beginnt wieder mit der y-Verschiebung.\newline\\
		
		\begin{minipage}{0.4\textwidth}
			\centering
			\includegraphics{img/pooling.png}
			\captionof{figure}{Maxpooling mit einem 2x2 Filter\cite{karpathy}}
			\label{img:pooling}
		\end{minipage}\newline
		
		\subsubsection*{Fully-Connected Layer} \\toDo
		Neuronen in einer FCL haben Verbindungen zu allen Knoten der vorherigen Schicht. Ihre Aktivierung wird durch eine Matrixmultiplikation und einem Bias-Offset berechnet \cite{karpathy}. Die FCL wird als Ausgabeschicht verwendet um aus der Eingangsmatrix einen Vektor zu erzeugen.
		
		\subsubsection{Training}
		Cross entropie
		
		\subsubsection{Optimierung}
		Dropout
		mini batch verfahren
		\subsection{}
		
		\subsection{Hybride Musikempfehlung mit einem Neuronalen Netzwerk}
		Im Unterschied zu der zuvor dargestellten Forschung (3.2) wird in der jetzigen ein Deep Belief Netzwerk(DBN) verwendet, um ein hybrides inhaltsbasiertes Musikempfehlungssystem zu entwickeln. Bisherige inhaltsbasiertes Systeme verfolgen typischerweise einem zweistufigen Ansatz: zunächst extrahieren sie aus Audioinhalte den MFCC Koeffizienten; anschließend prognostizieren sie Musikpräferenzen eines Nutzers. Das nachfolgende Modell führt dieses beiden Schritte simultan und automatisch aus. \cite{wang}\newline
		Das hybride Modell basiert auf einem hierarchischen linearen Modell mit einem Deep Belief Netzwerk(HLDBN), dass zunächst erläutert wird, um anschließend die Funktionsweise des hybriden Systems darzustellen.
		
		\begin{minipage}{0.45\textwidth}
			\centering
			\includegraphics{img/hlmdbn.png}
			\captionof{figure}{Hierarchisches lineares Modell eins Deep Belief Netzwerks \cite{wang}}
			\label{img:hlmdbn}
		\end{minipage}
	
		\subsubsection{Hierarchisches lineares Modell mit einem Deep Belief Netzwerk}
		Das in Abbildung \ref{img:hlmdbn} gezeigte Modell ist wie folgt definiert: \textit{f\textsubscript{v}} sind Musikmerkmale eines Liedes \textit{v}, die durch den Eigenvektor x\textsubscript{v} automatisch errechnet werden. Die bevorzugte Musik eines Benutzer \textit{u} wird as Vektor $\beta$\textsubscript{u} bezeichnet. $\Omega$ bezeichnet die Parameter, die das DBNs lernt. Die Bewertung, die \textit{u} einem Lied \textit{v} gibt, ist eine Skalarprodukt von \textit{r\textsubscript{xv}} und $\beta$\textsubscript{u}. Durch $\sigma$\textsubscript{R} wird die Varianz aller Bewertungen des Nutzers betrachtet. $\mu$ repräsentiert den allgemeinen Musikgeschmack aller Benutzer, wobei $\sigma$\textsubscript{u} die Varianz des einzelnen Nutzers definiert. Alle Benutzer und Lieder Paare werden als \textit{I} bezeichnet. Für eine Regularisierung der Werte wird die Gaußsche Normalverteilung $\mathcal{N}$ verwenden.\footnote[7]{$\mathcal{N}$(a,b) ist die Normalverteilung mit Mittelwert a und Varianz b. x $\sim$ p zeigt, dass x die Verteilung p erfüllt} \cite{wang}\newline Das Modell wird wie folgt formuliert:\newline\\
		\begin{minipage}{0.45\textwidth}
			\centering
			$\textit{r\textsubscript{xv}}\sim\mathcal{N}(\beta'\textit{x\textsubscript{v}},\sigma\textsuperscript{2}\textsubscript{R})$\\
			$\beta\sim\mathcal{N}(\mu,\sigma\textsuperscript{2}\textsubscript{u}\textit{I})$\\
			$\textit{x\textsubscript{v}} = DBN(\textit{f\textsubscript{v}};\Omega)$
		\end{minipage}\newline\\
		Für das Training des Systems wird die Maximum Likelihood-Funktion oder auch Cross-Entropy verwendet. Als Optimierungsmethode wird das stochastische Mini-Batch Verfahren genutzt, um ein Overfitting der Parameter zu vermeiden. Nach der Lernphase kann \textit{r\textsubscript{xv}} geschätzt werden, wodurch auch neue Lieder empfohlen werden können. \cite{wang}
		
		\subsubsection{Hybrides Modell mit einem Deep Belief Netzwerk}
		
		\begin{minipage}{0.45\textwidth}
			\centering
			\includegraphics{img/hybrid.png}
			\captionof{figure}{Hybrides Empfehlungs Modell \cite{wang}}
			\label{img:hybrid}
		\end{minipage}
		
		\section{Vergleich mit Stand der Forschung}
		
		\section{Diskussion der zukünftigen Forschungstrends}
		
		%\bibliographystyle{abbrvdin}
		\bibliographystyle{unsrt}
		\bibliography{lit}
		
	\end{multicols}
	
\end{document}
