\documentclass[twosided,a4,10pt]{article}
\usepackage[utf8]{inputenc}
\usepackage{amsmath}
\usepackage{amsfonts}
\usepackage{amssymb}
\usepackage{textcomp}
\usepackage{german}
\usepackage{graphicx}
\usepackage[usenames,dvipsnames]{xcolor}
\usepackage{pifont}
\usepackage{nicefrac}
\usepackage{sectsty}

% ------
% Fonts and typesetting settings
\usepackage[sc]{mathpazo}
\usepackage[T1]{fontenc}
\linespread{1.1} % Palatino needs more space between lines
\usepackage{microtype}
\subsectionfont{\fontsize{10}{15}\selectfont}

% ------
% Page layout
\usepackage[hmarginratio=1:1,top=32mm,columnsep=20pt]{geometry}
\usepackage[font=it]{caption}
\usepackage{paralist}
\usepackage{multicol}

% ------
% Abstract
\usepackage{abstract}
	\renewcommand{\abstractnamefont}{\normalfont\bfseries}
	\renewcommand{\abstracttextfont}{\normalfont\small\itshape}


% ------
% Titling (section/subsection)
\usepackage{titlesec}
%\renewcommand\thesection{\Roman{section}}
\titleformat{\section}[block]{\large\scshape\centering}{\thesection.}{1em}{}

% ------
% Clickable URLs (optional)
\usepackage{hyperref}

% ------
% Header/footer
\usepackage{fancyhdr}
	\pagestyle{fancy}
%	\fancyhead{}
%	\fancyfoot[C]{WIS WS 2017/18 $\cdot$
%          Software Engineering $\cdot$ Prof. Dr. Dünnweber}
	\fancyhead[C]{OTH Regensburg $\cdot$ Fakultät IM}
%	\fancyfoot[RO,LE]{\thepage}
	\fancyfoot[L]{WIS $\cdot$ WS 2017/18}
	\fancyfoot[R]{Prof. Dr. Dünnweber}
	\fancyfoot[C]{\thepage}


% ------
% Maketitle metadata
\title{\vspace{-5mm}%
	\fontsize{20pt}{10pt}\selectfont
	\textbf{Automatisierte Musikempfehlung mit Neuronalen Netzwerken}
	}	
\vspace{-5mm}\date{}
\author{
	\large
       \begin{minipage}[t]{0.5\linewidth}
         \begin{center}
           	\textsc{Weidhas Philipp}\\[2mm]
                 \normalsize	Matr.nr: 123456\\
                 \normalsize
                 \href{mailto:philipp.weidhas@st.oth-regensburg.de}
                 {philipp.weidhas@st.oth-regensburg.de}      
         \end{center}
       \end{minipage}        
       \begin{minipage}[t]{0.5\linewidth}
         \begin{center}
           	\textsc{Wildgruber Markus}\\[2mm]
                 \normalsize	Matr.nr: 123456\\
                 \normalsize
                 \href{mailto:markus.wildgruber@stud.oth-regensburg.de}
                 {markus.wildgruber@stud.oth-regensburg.de}      
         \end{center}
       \end{minipage}
     }




%%%%%%%%%%%%%%%%%%%%%%%%
\begin{document}

\maketitle
\thispagestyle{fancy}

	

\begin{multicols}{2}

\begin{abstract}
\noindent Hier kommt die Zusammenfassung...
\end{abstract}


\section{Einleitung}

Man kann auch ganz andere Gerte (Ha, der erste richtige Umlaut außer
Esszett!) referenzieren, zum Beispiel die fundamentale Gleichung Nummer
\ref{Gleichung}.
\begin{equation}\label{Gleichung}
\int^{\infty}_{-\infty} e^{-x^{2}}dx = \sqrt{\pi} 
\end{equation}

Es ist zu beachten, da \LaTeX-Befehle hufig Argumente haben. Diese stehen
dann in geschweiften Klammern nach dem Befehl. Zum Beispiel wurde "`das
\texttt{german}-Paket"' mittels "`\verb+das \texttt{german}-Paket+"'
gesetzt. Deutsche Gchen bekommt man rigens indem man \verb+"`Text"'+
eingibt. 

\section{Bestehende Ansätze zur Problemlösung}

\subsection{Inhaltsbasierter Filter}

\subsection{Kontextbasierter Filter}

\subsection{Hybrider Ansatz}
Text

\section{Tiefe Neuronale Netzwerke}
Nachdem Alex Krizhevsky mit seinem Team den ImageNet ILSVRc-2012 Kontest mit Hilfe eines tiefen Neuronalen Netzwerks (DNN) gewonnen hatte wurden DNNs neben dem Bereich der Bildklassifizierung \cite{alex}, auch in Gesichts- \cite{ding}, Spracherkennung \cite{graves} und der Inhaltsbasierten Musikempfehlung \cite{oord}. \newline Tiefe Neuronale Netzwerke imitieren die Architektur des menschlichen Gehirns \cite{wang}. Dabei können sie verschiedene Funktionen auf drei unterschiedliche Arten lernen. Dem überwachten Lernen (supervised learning) bei dem das DNN eine Eingabe erhält, dessen Ausgabe bekannt ist. Durch das Vergleichen der Netzwerkausgabe mit der Erwarteten, kann dann das DNN dementsprechend konfiguriert werden. Beim Unüberwachten Lernen (unsupervised learning) erhält das DNN verschiedene Eingaben und soll selbständig zusammenhänge zwischen diesen erkennen. Beim bestärkten Lernen (reinforcement learning) befindet sich das DNN in einer ihm unbekannter Umgebung, die es zu erforschen gilt. Gewünschtes Verhalten wird belohnt, wodurch es lernt die richtigen Entscheidungen zu treffen \cite{wang2}. Im folgenden Absatz wird eine Übersicht über den Aufbau und das Training von DNNs dargelegt. Anschließend werden verschiedene Ansätze der Inhaltbasierten Musikempfehlung miteinander verglichen.

\subsection{Aufbau eines tiefen Neuronalen Netzes}

\subsection{Vergleich verschiedener Ansätze}
Text

\section{Experiment}
Text

\subsection{Aufbau}
Text

\subsection{Ergebnis}
Text

\section{Vergleich mit Stand der Forschung und Ausblick}
Text

\bibliographystyle{abbrvdin}
\bibliography{lit}

\end{multicols}

\end{document}
