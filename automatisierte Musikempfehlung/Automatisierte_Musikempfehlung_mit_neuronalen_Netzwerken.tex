\documentclass[twosided,a4,10pt]{article}
\usepackage[utf8]{inputenc}
\usepackage{amsmath}
\usepackage{amsfonts}
\usepackage{amssymb}
\usepackage{textcomp}
\usepackage{german}
\usepackage{graphicx}
\usepackage[usenames,dvipsnames]{xcolor}
\usepackage{pifont}
\usepackage{nicefrac}
\usepackage{sectsty}

% ------
% Fonts and typesetting settings
\usepackage[sc]{mathpazo}
\usepackage[T1]{fontenc}
\linespread{1.1} % Palatino needs more space between lines
\usepackage{microtype}
\subsectionfont{\fontsize{10}{15}\selectfont}

% ------
% Page layout
\usepackage[hmarginratio=1:1,top=32mm,columnsep=20pt]{geometry}
\usepackage[font=it]{caption}
\usepackage{paralist}
\usepackage{multicol}

% ------
% Abstract
\usepackage{abstract}
	\renewcommand{\abstractnamefont}{\normalfont\bfseries}
	\renewcommand{\abstracttextfont}{\normalfont\small\itshape}


% ------
% Titling (section/subsection)
\usepackage{titlesec}
%\renewcommand\thesection{\Roman{section}}
\titleformat{\section}[block]{\large\scshape\centering}{\thesection.}{1em}{}

% ------
% Clickable URLs (optional)
\usepackage{hyperref}

% ------
% Header/footer
\usepackage{fancyhdr}
	\pagestyle{fancy}
%	\fancyhead{}
%	\fancyfoot[C]{WIS WS 2017/18 $\cdot$
%          Software Engineering $\cdot$ Prof. Dr. Dünnweber}
	\fancyhead[C]{OTH Regensburg $\cdot$ Fakultät IM}
%	\fancyfoot[RO,LE]{\thepage}
	\fancyfoot[L]{WIS $\cdot$ WS 2017/18}
	\fancyfoot[R]{Prof. Dr. Dünnweber}
	\fancyfoot[C]{\thepage}


% ------
% Maketitle metadata
\title{\vspace{-5mm}%
	\fontsize{20pt}{10pt}\selectfont
	\textbf{Automatisierte Musikempfehlung mit Neuronalen Netzwerken}
	}	
\vspace{-5mm}\date{}
\author{
	\large
       \begin{minipage}[t]{0.5\linewidth}
         \begin{center}
           	\textsc{Weidhas Philipp}\\[2mm]
                 \normalsize	Matr.nr: 123456\\
                 \normalsize
                 \href{mailto:philipp.weidhas@st.oth-regensburg.de}
                 {philipp.weidhas@st.oth-regensburg.de}      
         \end{center}
       \end{minipage}        
       \begin{minipage}[t]{0.5\linewidth}
         \begin{center}
           	\textsc{Wildgruber Markus}\\[2mm]
                 \normalsize	Matr.nr: 123456\\
                 \normalsize
                 \href{mailto:markus.wildgruber@stud.oth-regensburg.de}
                 {markus.wildgruber@stud.oth-regensburg.de}      
         \end{center}
       \end{minipage}
     }




%%%%%%%%%%%%%%%%%%%%%%%%
\begin{document}

\maketitle
\thispagestyle{fancy}

	

\begin{multicols}{2}

\begin{abstract}
\noindent Hier kommt die Zusammenfassung...
\end{abstract}


\section{Einleitung}

Man kann auch ganz andere Gerte (Ha, der erste richtige Umlaut außer
Esszett!) referenzieren, zum Beispiel die fundamentale Gleichung Nummer
\ref{Gleichung}.
\begin{equation}\label{Gleichung}
\int^{\infty}_{-\infty} e^{-x^{2}}dx = \sqrt{\pi} 
\end{equation}

Es ist zu beachten, da \LaTeX-Befehle hufig Argumente haben. Diese stehen
dann in geschweiften Klammern nach dem Befehl. Zum Beispiel wurde "`das
\texttt{german}-Paket"' mittels "`\verb+das \texttt{german}-Paket+"'
gesetzt. Deutsche Gchen bekommt man rigens indem man \verb+"`Text"'+
eingibt. 

\section{Bestehende Ansätze zur Problemlösung}


Jetzt wird noch kurz erlutert, wie das Programm
\textit{MakeIndex}\index{MakeIndex@\textit{MakeIndex}} dazu verwendet werden
kann, einen Stichwortverzeichnis zu erstellen. Das kann bei größeren Werken
für die Leser von nahezu unschtzbarem Wert sein. 
%%
Man muß dazu das Paket
\texttt{makeidx}\index{makeidx-Paket@\texttt{makeidx}-Paket} einbinden sowie
den Befehl \verb+\makeindex+ in der Prambel aufrufen. An die Stelle im
Dokument, wo das Stichwortverzeichnis erscheinen soll, kommt der
\verb+\printindex+-Befehl. Zum Erzeugen eines Indexeintrages für das Wort
Indexeintrag\index{Indexeintrag} muß nur \emph{direkt} hinter dem
entsprechenden Vorkommen des Wortes der Befehl \verb+\index{Indexeintrag}+
stehen.

Wenn die \LaTeX-Datei \texttt{mein-text.tex} heißt, muß man dann immer drei
Aufrufe machen:
\begin{enumerate}
	\item \texttt{latex mein-text}
	\item \texttt{makeindex mein-text}
	\item \texttt{latex mein-text}
\end{enumerate}
Wem der Standardtitel "`Index"' nicht behagt, der kann das auch zum Beispiel
mit einem \verb+\renewcommand{\indexname}{Stichworte}+ den eigenen Wünschen
anpassen. 

\subsection{Inhaltsbasierter Filter}
Und noch etwas Text... \cite{muster}

\subsection{Kontextbasierter Filter
	Text}

\subsection{Hybrider Ansatz}
Text

\section{Ansatz mit Hilfe Neuronale Netzwerke}
Text

\subsection{Funktion Neuronale Netze}
Text

\subsection{Vergleich verschiedener Ansätze}
Text

\section{Experiment}
Text

\subsection{Aufbau}
Text

\subsection{Ergebnis}
Text

\section{Vergleich mit Stand der Forschung und Ausblick}
Text

\bibliographystyle{abbrvdin}
\bibliography{lit}

\end{multicols}

\end{document}
